\pagenumbering{roman}

\begin{abstract}
\noindent
Over the last few years, advances in natural language processing (NLP) have enabled us to learn more from textual data. To this end, word embedding models learn vectorized representations of words by training on big sets of texts (e.g. the entire Wikipedia corpus). Word2vec is a word embedding model which learns single vector representations of words. However, by creating such single vector representations of words, it becomes hard to separate between word meanings, as the single vector representations have to cover all the word meanings. Words with multiple meanings are called \textit{polysemous}, and determining the word meanings is a challenging problem in NLP. Traditionally, word embeddings from word2vec are analyzed using analogy and cluster analysis. In analogy analysis of word embeddings, it is common to show relationships between words, e.g. that the relationship between \textit{king} and \textit{man} is the same as that between \textit{queen} and \textit{woman}, whereas, in cluster analysis of word embeddings, it is common to show how similar words cluster together, e.g. the clustering of country-related words. Moreover, due to recent developments in the field of topological data analysis, a topological measure of polysemy was introduced, which attempts to identify polysemous words from their word embeddings. The goal of this thesis is to show how word embeddings traditionally are analyzed using analogies and clustering algorithms and to use methods such as topological polysemy for identifying polysemous words of various word embeddings. Our results show that we are effectively able to cluster word embeddings into groups of varying sizes. Results also revealed that the measure of topological polysemy was inconsistent across word embeddings, and our proposed supervised models attempt to overcome and improve on this work.
\end{abstract}

\renewcommand{\abstractname}{Acknowledgements}
\begin{abstract}
I would like to thank all the people who helped and supported me with the work of this thesis.
\\\\
Firstly, I would like to thank my supervisor, Nello Blaser, for his outstanding guidance during the work of this thesis. Your expertise and insight have brought the quality of this work to a higher level, and I am very grateful for that.
\\\\
Secondly, I would like to thank the research group in machine learning at the Department of Informatics at UiB for providing me with computational resources. The computational resources have helped me performing the analyses in this thesis, which I would not have been able to at an equal scale otherwise.
\\\\
Thirdly, I would like to thank my fellow graduate students, particularly Naphat, for the long lunch breaks and the helpful academic discussions. 
\\\\
Lastly, I would like to thank my family and friends for all their love and support. Specifically, I would like to express my gratitude towards my father and Nora. You have always been there for me and have helped me to rest my mind outside the thesis.

\vspace{1cm}
\hspace*{\fill}\texttt{Jonas Folkvord Triki}\\ 
\hspace*{\fill} 01 June, 2021
\end{abstract}
\newpage